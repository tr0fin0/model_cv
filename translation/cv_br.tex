\documentclass[../cv.tex]{subfiles}

\begin{document}\selectlanguage{portuguese}
\columnratio{\columnsize}
\begin{paracol}{2}
    \maketitle

    \section*{EXPERIÊNCIAS}
    \topicblock{
        Estagiário de Engenharia de Garantia da Qualidade
    }{
        \href{https://www.ansys.com/}{\myuline{Ansys}}, \href{https://www.synopsys.com/}{\myuline{Synopsys}}
    }{
        La Farlède, França
    }{
        6 meses: 04/2025 -- 10/2025
    }{
        Em colaboração com o time de Qualidade da Ansys Optics, eu ajudei a melhorar o processo de validação por meio da \textbf{conversão de testes unitários manuais em automáticos} usando gRPC, IronPython e Python, economizando cerca de 20 horas de trabalho por versão de software. Eu também conduzi a pesquisa e desenvolvimento de uma melhor \textbf{função multivariada de comparação de image} em Python permitindo completa automatização de processos.   
    }{
        gRPC, Image Comparison e Python.
    }

    \topicblock{
        Estagiário de Designer de Hardware Embarcado
    }{
        \href{https://www.sysnav.fr/}{\myuline{Sysnav}}
    }{
        Vernon, França
    }{
        5 meses: 04/2024 -- 09/2024
    }{
        Em colaboração com um engenheiro eletrônico, eu ajudei a cria uma ferramente de teste para o produto Syde, permitindo a criação de cenários de utilização através de uma \textbf{biblioteca Python} que se comunicava por UART com um microprocessador STM32H7 em C. Isso permitiu a simulação de cenários de forma consistente, escalando o \textbf{controle de qualidade} na saída da produção.
    }{
        C, Git e Python.
    }

    \topicblock{
        Estagiário de Gerente de Fornecedores de Varejo
    }{
        \href{https://www.aboutamazon.fr/}{\myuline{Amazon}}
    }{
        Clichy, França
    }{
        6 meses: 09-2023 -- 03/2024
    }{
        Em colaboração com um Gerente Senior de Fornecedores, ajudei a fornecer duas ferramentas para otimizar o desempenho dos negócios e impulsionar o crescimento das vendas: uma macro do Excel para \textbf{identificar a seleção de licenças vinte vezes mais rápido}, permitindo análises profundas em grandes conjuntos de dados, descobrindo tendências e tomando medidas; e um \textbf{painel abrangente e visualmente atraente} no Excel, facilitando a comparação em vários níveis de granularidade, simplificando a identificação de oportunidades de crescimento e a implementação de novas estratégias.
    }{
        Excel, SQL e VBA.
    }

    \topicblock{
        Estagiário de Pesquisa em Neurociência
    }{
        \href{https://www.isir.upmc.fr/}{\myuline{ISIR}}, \href{https://www.sorbonne-universite.fr/}{\myuline{Sorbonne Université}}
    }{
        Paris, França
    }{
        3 meses: 05/2023 -- 08/2023
    }{
        Desenvolver um ambiente de simulação amigável usando Webots e Python, facilitando a avaliação da \textbf{modelagem sensório-motora} do hipocampo com base na estrutura proposta por Benoît Girard e Sylvain Argentieri.
    }{
        Git, Python e Webots
    }

    \topicblock{
        Chefe de Equipe
    }{
        \href{https://www.linkedin.com/company/equipe-phoenix/}{\myuline{Equipe Phoenix de Robótica da UNICAMP}}
    }{
        Campinas, Brasil
    }{
        1 ano: 07/2021 -- 07/2022
    }{
        Melhorar o processo de seleção dos membros da equipe, visando mais diversidade, resultando na melhor taxa de inscrição, mais de 125 pessoas de 500, e resultando na \textbf{melhor porcentagem de mulheres} entre os aprovados, 16 de 35 pessoas.
    }{
        Comunicação, Liderança e Gestão de Projetos/Produtos.
    }

    \section*{FORMAÇÃO}
    \topicblock{
        Engenharia da Computação em Robótica, Mestrado
    }{
        \href{https://www.ensta-paris.fr/}{\myuline{ENSTA Paris}}, \href{https://www.ip-paris.fr/}{\myuline{Institut Polytechnique de Paris}}
    }{
        Palaiseau, França
    }{
        3 anos: 2022 -- 2025
    }{
        Aprendizado de máquina, análise e indexação de imagens, controle de sistemas, FPGA Xilinx, filtro de Kalman, engenharia de software, programação paralela, visão, Vivado HLS / SDK.
    }{
        Cursos extras de pesquisa e inovação na \href{https://www.polytechnique.edu/}{École Polytechnique} durante o último ano.
    }
    \topicblock{
        Engenharia de Controle e Automação, Bacharelado
    }{
        \href{https://www.fem.unicamp.br/index.php/pt-br/engenharia-mecanica}{\myuline{FEM}}, \href{https://www.unicamp.br/unicamp/universidade}{\myuline{UNICAMP}}
    }{
        Campinas, Brasil
    }{
        6 anos: 2019 -- presente
    }{}{
        Top 7\% da classe em duplo diploma com a ENSTA Paris.
    }

    \switchcolumn

    \begin{figure}[h]
        \centering
        \includegraphics[width=\linewidth]{images/profile.jpg}
    \end{figure}

    \presentation{
        Sou brasileiro, tenho 24 anos e cidadania portuguesa, com graduação em engenharia de controle e automação pela UNICAMP e mestrado em engenharia de robótica pela ENSTA Paris.\\\\
        Estou buscando uma \textbf{vaga em tempo integral} na área de desenvolvimento de software, idealmente com envolvimento em hardware, disponível para começar a partir de agosto de 2026 e com disponibilidade para trabalhar em outros países.
    }
    \vspace{5mm}

    \section*{LANGUAGES}
    \topicline{Português}{Native}
    \colorbar{1}{45*1.0}
    \topicline{Francês}{Fluente\hfill\textbf{C2} (\href{https://www.coe.int/en/web/common-european-framework-reference-languages/level-descriptions}{\myuline{CEFR}})}
    \colorbar{1}{45*1.0}
    \topicline{Inglês}{Fluente\hfill\textbf{C2} (\href{https://www.coe.int/en/web/common-european-framework-reference-languages/level-descriptions}{\myuline{CEFR}})}
    \colorbar{1}{45*1.0}

    \topicline{Espanhol}{Intermediario\hfill\textbf{B1} (\href{https://www.coe.int/en/web/common-european-framework-reference-languages/level-descriptions}{\myuline{CEFR}})}
    \colorbar{1}{45*0.5}

    \section*{HABILIDADES}
    \noindent\textbf{Programação:}
    \vspace{-2.5mm}\begin{table}[h]
        \centering
        \begin{tabular}{lll}
            \textcolor{text_color}{Assembly} &
            \textcolor{text_color}{Bash} &
            \textcolor{text_color}{C/C++}
            \\
            \textcolor{text_color}{CI/CD} &
            \textcolor{text_color}{IronPython} &
            \textcolor{text_color}{LaTeX}
            \\
            \textcolor{text_color}{Python} &
            \textcolor{text_color}{VBA} &
            \textcolor{text_color}{VHDL}
            \\
        \end{tabular}
    \end{table}

    \topicline{Ferramentas}{Git, Docker, VSCode e WSL}

    \topicline{Adobe}{Illustrator e Photoshop}
    \topicline{Eletrônica}{Altium Designer}
    \topicline{Office}{Excel, PowerPoint e Word}

    \topicline{Softskills}{Comunicação, Liderança e Gestão de Projetos/Produtos}

    \section*{VOLUNTARIADO}
    \topicblock{
        \href{https://openbadgefactory.com/v1/assertion/e93057f5ecd084fbfba3636a1393600d98ce97fe.html}{\myuline{Voluntário Paraolímpiada}}
    }{
        \href{https://olympics.com/fr/paris-2024}{\myuline{Paris 2024}}
    }{
        Villepinte, França
    }{
        2024
    }{}{}

    \topicblock{
        Embaixador
    }{
        \href{https://www.patronos.org/}{\myuline{Patronos UNICAMP}}
    }{
        Campinas, Brasil
    }{
        2021
    }{}{}
\end{paracol}
\end{document}
