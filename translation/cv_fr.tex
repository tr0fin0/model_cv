\documentclass[../cv.tex]{subfiles}

\begin{document}
\columnratio{\columnsize}
\begin{paracol}{2}
    \maketitle

    \section*{EXPÉRIENCES}
    \topicblock{Embedded Hardware Designer Intern}{\href{https://www.sysnav.fr/}{\myuline{Sysnav}}}{Vernon, France}{5 mois: 04/2024 -- 09/2024}{En collaboration avec un ingénieur électronique, j'ai aidé à concevoir 1 outil de test pour le produit Syde, permettant la création de scénarios d'utilisation grâce à une \textbf{bibliothèque créée en Python} qui communiquait via UART avec un STM32H7 programmé en C.\\

    Cela a permis de simuler différents scénarios de manière \textbf{automatisée}, en mettant à l'échelle le processus de \textbf{contrôle de qualité} après la production du produit.}{C, Git et Python.}

    \topicblock{Retail Vendor Manager Intern}{\href{https://www.aboutamazon.fr/}{\myuline{Amazon}}}{Clichy, France}{6 mois: 09-2023 -- 03/2024}{En collaboration avec un Senior Vendor Manager, j'ai aidé à fournir 2 outils conçus pour optimiser les performances commerciales et stimuler la croissance des ventes. Cela incluait un macro Excel conçu pour \textbf{identifier la sélection sous licence vingt fois plus vite}, permettant des analyses approfondies dans de grands ensembles de données, découvrant des tendances et extrayant des insights clients.\\

    J'ai également développé un \textbf{dashboard complet et visuellement attrayant} dans Excel, facilitant le benchmarking à différents niveaux de granularité, simplifiant l'identification des opportunités de croissance et la mise en œuvre de nouvelles stratégies pour l'équipe.}{Excel, SQL et VBA.}

    \topicblock{Neuroscience Research Intern}{\href{https://www.isir.upmc.fr/}{\myuline{ISIR}}, \href{https://www.sorbonne-universite.fr/}{\myuline{Sorbonne Université}}}{Paris, France}{3 mois: 05/2023 -- 08/2023}{Développement d'un environnement de simulation utilisant Webots et Python, facilitant l'évaluation de la \textbf{modélisation sensorimotrice} de l'hippocampe basée sur le cadre proposé par Benoît Girard et Sylvain Argentieri}{Git, Python et Webots.}

    \topicblock{Head of Team}{\href{https://www.linkedin.com/company/equipe-phoenix/}{\myuline{Phoenix Robotics Team of UNICAMP}}}{Campinas, Brésil}{1 an: 07/2021 -- 07/2022}{Amélioration du processus de sélection des membres de l'équipe, visant une plus grande diversité, ce qui a abouti au meilleur ratio d'inscription, plus de 125 personnes sur 500, et au \textbf{meilleur pourcentage de femmes} parmi les approuvés, 16 sur 35 personnes.}{Communication, Leadership et Gestion de Projet / Produit.}

    \section*{ÉDUCATION}
    \topicblock{Auditeur au Master Systèmes Embarqués et Traitement de l'Information}{\href{https://www.universite-paris-saclay.fr/}{\myuline{Université Paris Saclay}}}{Gif-sur-Yvette, France}{2024 -- Présent}{Embedded Linux, Fusion de données multicapteurs et raisonnment sour incertitudes, Ordonnancement et noyaix pour les systèmes embarqués temps réel, Vision robotique.}{}
    \topicblock{Diplôme d'Ingénieur Informatique, MSc Robotique}{\href{https://www.ensta-paris.fr/}{\myuline{ENSTA Paris}}, \href{https://www.ip-paris.fr/}{\myuline{Institut Polytechnique de Paris}}}{Palaiseau, France}{2 an: 2022 -- Présent}{Apprentissage automatique, Analyse et indexation d'images, Deep Learning, Dynamique et contrôle des systèmes, Filtre de Kalman, FPGA Xilinx, Génie logiciel, Navegation Robotique, Programmation Parallèles, Vision 3D, Vivado HLS / SDK.}{}
    \topicblock{Ingénierie de Contrôle et d'Automatisation, BSc Engineering}{\href{https://www.fem.unicamp.br/index.php/pt-br/engenharia-mecanica}{\myuline{FEM}}, \href{https://www.unicamp.br/unicamp/universidade}{\myuline{UNICAMP}}}{Campinas, Brésil}{5 an: 2019 -- Présent}{}{Top 7\% de classe en double diplôme à l'ENSTA Paris.}

    \switchcolumn

    \begin{figure}[h]
        \centering
        \includegraphics[width=\linewidth]{tpack/cv/images/profile_social_square.jpg}
    \end{figure}

    \presentation{Étudiant ingénieur à l'ENSTA Paris en BAC+5, je suis actuellement à la recherche d'un \textbf{stage de fin d'études} en développement logiciel et/ou matériel pour:
    \begin{itemize}[noitemsep, leftmargin=5mm]
        \item l'analyse de données;
        \item la microélectronique;
        \item les systèmes embarqués;
    \end{itemize}
    à partir d'avril de 2025, pour une durée de 6 mois.}
    \vspace{5mm}

    \section*{LANGUES}
    \topicline{Portugais}{Langue Maternelle}
    \colorbar{1}{45*1.0}

    \topicline{Français}{Avancé\hfill\textbf{C1} (\href{https://www.coe.int/en/web/common-european-framework-reference-languages/level-descriptions}{\myuline{CEFR}})}
    \colorbar{1}{45*0.8}
    \topicline{Anglais}{Avancé\hfill\textbf{C1} (\href{https://www.coe.int/en/web/common-european-framework-reference-languages/level-descriptions}{\myuline{CEFR}})}
    \colorbar{1}{45*0.8}
    \topicline{Espagnol}{Débutant\hfill\textbf{A2} (\href{https://www.coe.int/en/web/common-european-framework-reference-languages/level-descriptions}{\myuline{CEFR}})}
    \colorbar{1}{45*0.25}


    \section*{COMPÉTENCES}
    \noindent\textbf{Programmation:}
    \vspace{-2.5mm}\begin{table}[h]
        \centering
        \begin{tabular}{lll}
            \textcolor{text_color}{Assembly} & \textcolor{text_color}{C/C++} & \textcolor{text_color}{LaTeX}\\
            \textcolor{text_color}{Bash} & \textcolor{text_color}{Java} & \textcolor{text_color}{MATLAB}\\
            \textcolor{text_color}{VHDL} & \textcolor{text_color}{Python} & \textcolor{text_color}{VBA}\\
        \end{tabular}
    \end{table}

    \topicline{Outils}{Git, Docker, VSCode et WSL}

    \topicline{Adobe}{Illustrator et Photoshop}
    \topicline{Électronique}{Altium Designer}
    \topicline{Office}{Excel, PowerPoint et Word}

    \topicline{Softskills}{Communication, Leadership et Gestion de Projet / Produit}

    \section*{BÉNÉVOLE}
    \topicblock{\href{https://openbadgefactory.com/v1/assertion/e93057f5ecd084fbfba3636a1393600d98ce97fe.html}{\myuline{Volontaire Jeux Paralympiques}}}{\href{https://olympics.com/fr/paris-2024}{\myuline{Paris 2024}}}{Villepinte, France}{2024}{}{}

    \topicblock{Ambassadeur}{\href{https://www.patronos.org/}{\myuline{Patronos UNICAMP}}}{Campinas, Brésil}{2021}{}{}
\end{paracol}
\end{document}