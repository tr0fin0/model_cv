\documentclass[../cv.tex]{subfiles}

\begin{document}
\columnratio{\columnsize}
\begin{paracol}{2}
    \maketitle

    \section*{EXPÉRIENCES}
    \topicblock{Embedded Hardware Designer Intern}{\href{https://www.sysnav.fr/}{Sysnav}}{Vernon, France}{04/2024 -- 09/2024}{En collaboration avec un ingénieur électronique, j'ai aidé à concevoir un outil de test pour le produit Syde, permettant la création de scénarios d'utilisation grâce à une bibliothèque créée en Python qui communiquait via UART avec un STM32H7 programmé en C.\\

    Cela a permis de simuler différents scénarios de manière automatisée, en mettant à l'échelle le processus de contrôle de qualité après la production du produit.}{C, Git et Python.}

    \topicblock{Retail Vendor Manager Intern}{\href{https://www.aboutamazon.fr/}{Amazon}}{Clichy, France}{09-2023 -- 03/2024}{En collaboration avec un Senior Vendor Manager, j'ai aidé à fournir des outils conçus pour optimiser les performances commerciales et stimuler la croissance des ventes. Cela incluait un macro Excel conçu pour identifier la sélection sous licence vingt fois plus vite, permettant des analyses approfondies dans de grands ensembles de données, découvrant des tendances et extrayant des insights clients.\\

    J'ai également développé un dashboard complet et visuellement attrayant dans Excel, facilitant le benchmarking à différents niveaux de granularité, simplifiant l'identification des opportunités de croissance et la mise en œuvre de nouvelles stratégies pour l'équipe.}{Excel, PowerPoint et VBA.}

    \topicblock{Neuroscience Research Intern}{\href{https://www.isir.upmc.fr/}{ISIR}, \href{https://www.sorbonne-universite.fr/}{Sorbonne Université}}{Paris, France}{05/2023 -- 08/2023}{Développement d'un environnement de simulation utilisant Webots et Python, facilitant l'évaluation de la modélisation sensorimotrice de l'hippocampe basée sur le cadre proposé par Benoît Girard et Sylvain Argentieri}{Git et Python.}

    \topicblock{Head of Team}{\href{https://www.linkedin.com/company/equipe-phoenix/}{Phoenix Robotics Team of UNICAMP}}{Campinas, Brésil}{07/2021 -- 07/2022}{Amélioration du processus de sélection des membres de l'équipe, visant une plus grande diversité, ce qui a abouti au meilleur ratio d'inscription, plus de 125 personnes sur 500, et au meilleur pourcentage de femmes parmi les approuvés, 16 sur 35 personnes.}{Communication, Leadership et Gestion de Projet / Produit.}

    \section*{ÉDUCATION}
    \topicblock{Auditeur au Master Systèmes Embarqués et Traitemen de l'Information}{\href{https://www.universite-paris-saclay.fr/}{Université Paris Saclay}}{Gif-sur-Yvette, France}{09/2024 -- Présent}{Embedded Linux, Fusion de données multicapteurs et raisonnment sour incertitudes, Ordonnancement et noyaix pour les systèmes embarqués temps réel et Vision robotique.}{}
    \topicblock{Diplôme d'Ingénieur Informatique, MSc Robotique}{\href{https://www.ensta-paris.fr/}{ENSTA Paris}, \href{https://www.ip-paris.fr/}{Institut Polytechnique de Paris}}{Palaiseau, France}{07/2022 -- Présent}{Apprentissage automatique, Analyse et indexation d'images, Deep Learning, Dynamique et contrôle des systèmes, Filtre de Kalman, Génie logiciel, Programmation Parallèles et Segmentation / Vision 3D.}{}
    \topicblock{Ingénierie de Contrôle et d'Automatisation, BSc Engineering}{\href{https://www.fem.unicamp.br/index.php/pt-br/engenharia-mecanica}{FEM}, \href{https://www.unicamp.br/unicamp/universidade}{UNICAMP}}{Campinas, Brésil}{03/2019 -- Présent}{}{Double Diplôme avec ENSTA Paris}

    \switchcolumn

    \begin{figure}[h]
        \centering
        \includegraphics[width=\linewidth]{tpack/cv/images/profile_social_square.jpg}
    \end{figure}

    \presentation{Étudiant ingénieur à l'ENSTA Paris en BAC+5, je suis actuellement à la recherche d'un \textbf{stage de fin d'études} en développement logiciel et/ou matériel pour:
    \begin{itemize}[noitemsep, leftmargin=5mm]
        \item l'analyse de données;
        \item les systèmes embarqués;
        \item la robotique;
    \end{itemize}
    à partir d'avril de 2025, pour une durée de 6 mois.}
    \vspace{5mm}

    \section*{LANGUES}
    \topicline{Portugais}{Langue Maternelle}

    \topicline{Anglais}{Avancé\hfill\textbf{C1} (\href{https://www.coe.int/en/web/common-european-framework-reference-languages/level-descriptions}{CEFR})}
    \topicline{Français}{Avancé\hfill\textbf{C1} (\href{https://www.coe.int/en/web/common-european-framework-reference-languages/level-descriptions}{CEFR})}
    \topicline{Espagnol}{Débutant\hfill\textbf{A2} (\href{https://www.coe.int/en/web/common-european-framework-reference-languages/level-descriptions}{CEFR})}


    \section*{COMPÉTENCES}
    \noindent\textbf{Programmation:}
    \vspace{-2.5mm}\begin{table}[h]
        \centering
        \begin{tabular}{lll}
            \textcolor{text_color}{Assembly} & \textcolor{text_color}{C/C++} & \textcolor{text_color}{LaTeX}\\
            \textcolor{text_color}{Bash} & \textcolor{text_color}{Java} & \textcolor{text_color}{MATLAB}\\
            \textcolor{text_color}{VHDL} & \textcolor{text_color}{Python} & \textcolor{text_color}{VBA}\\
        \end{tabular}
    \end{table}

    \topicline{Outils}{Git, Docker, VSCode et WSL}

    \topicline{Adobe}{Illustrator et Photoshop}
    \topicline{Électronique}{Altium Designer}
    \topicline{Office}{Excel, PowerPoint et Word}

    \topicline{Softskills}{Communication, Leadership et Gestion de Projet / Produit}

    \section*{BÉNÉVOLE}
    \topicblock{\href{https://openbadgefactory.com/v1/assertion/e93057f5ecd084fbfba3636a1393600d98ce97fe.html}{Volontaire Jeux Paralympiques}}{\href{https://olympics.com/fr/paris-2024}{Paris 2024}}{Villepinte, France}{2024}{}{}

    \division

    \topicblock{Ambassadeur}{\href{https://www.patronos.org/}{Patronos UNICAMP}}{Campinas, Brésil}{2021}{}{}
\end{paracol}
\end{document}